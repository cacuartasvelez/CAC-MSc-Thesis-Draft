\chapter[OCT: Formulación y tratamiento matemático]{Tomografía óptica coherente: formulación y tratamiento matemático}

La OCT tiene diferentes vertientes, como se expresó anteriormente, una de ellas proviene del uso de interferómetros de luz blanca, para entender desde un punto de vista matemático lo que sucede en OCT, se comenzará por abordar la OCT en el dominio temporal.

\subsection{Tomografía óptica coherente en el dominio temporal}

\textbf{Intererometría con una fuente monocromática}. En la Sección~\ref{sec:OCT_Esquema} OCT se basa en un interferómetro de Michelson, como se muestra en la Fig. (Hay que hacerla). Comencemos con una onda plana compleja proveniente de la fuente $so$, propagándose en dirección $z$, $E_{so}$, de la forma:

\begin{equation}
	E_{so} = E_0 e^{-ik(\omega)z},
\end{equation}
\noindent donde $E_0$ es la amplitud de la onda, $i$ es la unidad imaginaria y $k$ es el número de onda $k = 2\pi /\lambda$. Para la entrada del interferómetro, se puede suponer que la intensidad está dividida en dos haces que combinados son $E_{so}$, de forma que

\begin{align}
	E_r = 1/\sqrt{2} E_{so} \notag \\
	E_s = i/\sqrt{2} E_{so}
\end{align}

